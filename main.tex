\documentclass[10pt,a4paper]{article}
%\usepackage{zh_CN-Adobefonts_external}
\usepackage{xeCJK}
\usepackage{amsmath, amsthm}
\usepackage{listings,xcolor}
\usepackage{geometry} % 设置页边距
\usepackage{fontspec}
\usepackage{graphicx}
\usepackage[colorlinks]{hyperref}
\usepackage{setspace}
\usepackage{fancyhdr} % 自定义页眉页脚
\usepackage{amsfonts}


\setsansfont{Consolas} % 设置英文字体
\setmonofont[Mapping={}]{Consolas} % 英文引号之类的正常显示,相当于设置英文字体

\linespread{1.2}

\title{XCPC Templates}
\author{Khoray}
\definecolor{dkgreen}{rgb}{0,0.6,0}
\definecolor{gray}{rgb}{0.5,0.5,0.5}
\definecolor{mauve}{rgb}{0.58,0,0.82}

\pagestyle{fancy}

\lhead{\CJKfamily{kai} Chongqing University} %以下分别为左中右的页眉和页脚
\chead{}

\rhead{\CJKfamily{kai} 第 \thepage 页}
\lfoot{} 
\cfoot{\thepage}
\rfoot{}
\renewcommand{\headrulewidth}{0.4pt} 
\renewcommand{\footrulewidth}{0.4pt}
%\geometry{left=2.5cm,right=3cm,top=2.5cm,bottom=2.5cm} % 页边距
\geometry{left=3.18cm,right=3.18cm,top=2.54cm,bottom=2.54cm}
\setlength{\columnsep}{30pt}

\makeatletter

\makeatother



\lstset{
    language    = c++,
    numbers     = left,
    numberstyle={                               % 设置行号格式
        \small
        \color{black}
        \fontspec{Consolas}
    },
	commentstyle = \color[RGB]{0,128,0}\bfseries, %代码注释的颜色
	keywordstyle={                              % 设置关键字格式
        \color[RGB]{40,40,255}
        \fontspec{Consolas Bold}
        \bfseries
    },
	stringstyle={                               % 设置字符串格式
        \color[RGB]{128,0,0}
        \fontspec{Consolas}
        \bfseries
    },
	basicstyle={                                % 设置代码格式
        \fontspec{Consolas}
        \small\ttfamily
    },
	emphstyle=\color[RGB]{112,64,160},          % 设置强调字格式
    breaklines=true,                            % 设置自动换行
    tabsize     = 4,
    frame       = single,%主题
    columns     = fullflexible,
    rulesepcolor = \color{red!20!green!20!blue!20}, %设置边框的颜色
    showstringspaces = false, %不显示代码字符串中间的空格标记
	escapeinside={\%*}{*)},
}

\begin{document}
\title{XCPC Templates}
\author {Khoray}
\maketitle
\newpage
\tableofcontents
\newpage

\section{数学}
%--------------------------------------------------------------------
\subsection{欧拉筛/积性函数筛/线性筛}
\begin{itemize}
    \item 积性函数筛,$f(pq)=f(p)f(q),(p,q)=1$
    \item calc\_f(val, power) 返回 $f(val^{power})$
\end{itemize}
\lstinputlisting{code/欧拉筛.cpp}
%--------------------------------------------------------------------
\subsection{快速幂}
\lstinputlisting{code/ksm.cpp}

%--------------------------------------------------------------------
\subsection{组合数}
\subsubsection{暴力}
\begin{itemize}
    \item 暴力求组合数 $\binom{n}{k}$, 时间复杂度 $O(\min(k, n - k))$.
    \item 前置:快速幂
    \item 模数必须是质数!
\end{itemize}
\lstinputlisting{code/暴力组合数.cpp}
%--------------------------------------------------------------------
\subsubsection{递推}
\begin{itemize}
    \item $O(n^2)$ 递推求,模数随意。
\end{itemize}
\lstinputlisting[]{code/递推组合数.cpp}
%--------------------------------------------------------------------
\subsubsection{逆元}
\begin{itemize}
    \item 模数必须是质数!
    \item 前置:快速幂
\end{itemize}
\lstinputlisting{code/逆元组合数.cpp}
%--------------------------------------------------------------------
\subsection{ExGCD}
\begin{itemize}
    \item 求解 $ax+by=(a,b)$ 的特解 $x_0, y_0$.
    \item 通解$x^* = x_0+\frac{bk}{(a,b)}, y^* = y_0-\frac{ak}{(a,b)}$ ( $k \in \mathbb{Z}$ ) .
\end{itemize}
\lstinputlisting{code/ExGCD.cpp}
%--------------------------------------------------------------------
\subsection{拉格朗日插值}
\begin{itemize}
    \item $f(x)$是多项式,并且我们知道一系列连续的点值 $f(l),\cdots,f(r)$ ,求解 $f(n)$。 $O(r-l)$
    \item 前置:逆元组合数
    \item 模数为质数
\end{itemize}
\lstinputlisting{code/拉格朗日插值.cpp}

%--------------------------------------------------------------------
\subsection{原根}
若 $g$ 满足:
\begin{equation*}
    \begin{aligned}
        (g,m)=1\\
        \delta_m(g)=\varphi(m)
    \end{aligned}
\end{equation*}
则 $g$ 为 $m$ 的原根。
找所有原根:
\begin{enumerate}
    \item 找到最小的原根 $g$。如果一个数 $g$ 是原根,那么 $\forall p|\varphi(m):g^{\dfrac{\varphi(m)}{p}}\neq1$
    \item 找小于 $m$ 与 $\varphi(m)$ 互质的数 $k$ ,则 $g^k$ 也是原根(能覆盖所有原根)个数为 $\varphi(\varphi(m))$ 个。
\end{enumerate}
题目:找出 $n$ 的所有原根,间隔 $d$ 输出。
\lstinputlisting{code/原根.cpp}
%--------------------------------------------------------------------
%--------------------------------------------------------------------
%--------------------------------------------------------------------


%--------------------------------------------------------------------
\subsection{多项式全家桶}
\begin{itemize}
    \item 注意调整原根g,模数mod,N开3到4倍数据范围,附录A
    \item 注意resize()
    \item 注意Inv/Ln的时候常数项不能为0
    \item 注意Exp的时候常数项必须是0
    \item 注意这里面的 ksm() 第三个参数是初值而不是模数
\end{itemize}
\lstinputlisting{code/NTT.cpp}



\end{document}