\documentclass[10pt,a4paper]{article}
%\usepackage{zh_CN-Adobefonts_external}
\usepackage{xeCJK}
\usepackage{amsmath, amsthm}
\usepackage{amssymb}
\usepackage{listings,xcolor}
\usepackage{geometry} % 设置页边距
\usepackage{fontspec}
\usepackage{graphicx}
\usepackage[colorlinks]{hyperref}
\usepackage{setspace}
\usepackage{fancyhdr} % 自定义页眉页脚
\usepackage{amsfonts}




\setsansfont{Consolas} % 设置英文字体
\setmonofont[Mapping={}]{Consolas} % 英文引号之类的正常显示,相当于设置英文字体

\linespread{1.2}

\title{C++期末复习资料}
\author{Khoray}
\definecolor{dkgreen}{rgb}{0,0.6,0}
\definecolor{gray}{rgb}{0.5,0.5,0.5}
\definecolor{mauve}{rgb}{0.58,0,0.82}

\pagestyle{fancy}

\lhead{\CJKfamily{kai} Chongqing University} %以下分别为左中右的页眉和页脚
\chead{}

\rhead{\CJKfamily{kai} 第 \thepage 页}
\lfoot{} 
\cfoot{\thepage}
\rfoot{}
\renewcommand{\headrulewidth}{0.4pt} 
\renewcommand{\footrulewidth}{0.4pt}
%\geometry{left=2.5cm,right=3cm,top=2.5cm,bottom=2.5cm} % 页边距
\geometry{left=3.18cm,right=3.18cm,top=2.54cm,bottom=2.54cm}
\setlength{\columnsep}{30pt}

\makeatletter

\makeatother



\lstset{
    language    = c++,
    numbers     = left,
    numberstyle={                               % 设置行号格式
        \small
        \color{black}
        \fontspec{Consolas}
    },
	commentstyle = \color[RGB]{0,128,0}\bfseries, %代码注释的颜色
	keywordstyle={                              % 设置关键字格式
        \color[RGB]{40,40,255}
        \fontspec{Consolas Bold}
        \bfseries
    },
	stringstyle={                               % 设置字符串格式
        \color[RGB]{128,0,0}
        \fontspec{Consolas}
        \bfseries
    },
	basicstyle={                                % 设置代码格式
        \fontspec{Consolas}
        \small\ttfamily
    },
	emphstyle=\color[RGB]{112,64,160},          % 设置强调字格式
    breaklines=true,                            % 设置自动换行
    tabsize     = 4,
    frame       = single,%主题
    columns     = fullflexible,
    rulesepcolor = \color{red!20!green!20!blue!20}, %设置边框的颜色
    showstringspaces = false, %不显示代码字符串中间的空格标记
	escapeinside={\%*}{*)},
}

\begin{document}
\title{C++期末复习资料1}
\author {Khoray}
\maketitle
\newpage
\tableofcontents
\newpage

\section{Data Structure}

\subsection{}

\subsection{线段树}
\begin{itemize}
    \item 维护的信息 Data 满足幺半群性质(有幺元,结合,封闭)。
    \item Mapping 是区间修改操作,必须满足可以进行 Composition(用于合并 Lazy 标记),Mapping 需要存在幺元。
\end{itemize}


\lstinputlisting{code/ds/线段树.cpp}

\subsection{珂朵莉树}

\newpage
\section{Math}
%--------------------------------------------------------------------
\subsection{欧拉筛/积性函数筛/线性筛}
\begin{itemize}
    \item 积性函数筛,$f(pq)=f(p)f(q),(p,q)=1$
    \item calc\_f(val, power) 返回 $f(val^{power})$
\end{itemize}
\lstinputlisting{code/math/欧拉筛.cpp}
%--------------------------------------------------------------------
\subsection{快速幂}
\lstinputlisting{code/math/ksm.cpp}

%--------------------------------------------------------------------
\subsection{组合数}
\subsubsection{暴力}
\begin{itemize}
    \item 暴力求组合数 $\binom{n}{k}$, 时间复杂度 $O(\min(k, n - k))$.
    \item 前置:快速幂
    \item 模数必须是质数!
\end{itemize}
\lstinputlisting{code/math/暴力组合数.cpp}
%--------------------------------------------------------------------
\subsubsection{递推}
\begin{itemize}
    \item $O(n^2)$ 递推求,模数随意。
\end{itemize}
\lstinputlisting[]{code/math/递推组合数.cpp}
%--------------------------------------------------------------------
\subsubsection{逆元}
\begin{itemize}
    \item 模数必须是质数!
    \item 前置:快速幂
\end{itemize}
\lstinputlisting{code/math/逆元组合数.cpp}
%--------------------------------------------------------------------
\subsection{ExGCD}
\begin{itemize}
    \item 求解 $ax+by=(a,b)$ 的特解 $x_0, y_0$.
    \item 通解$x^* = x_0+\frac{bk}{(a,b)}, y^* = y_0-\frac{ak}{(a,b)}$ ( $k \in \mathbb{Z}$ ) .
\end{itemize}
\lstinputlisting{code/math/ExGCD.cpp}
%--------------------------------------------------------------------
\subsection{拉格朗日插值}
\begin{itemize}
    \item $f(x)$是多项式,并且我们知道一系列连续的点值 $f(l),\cdots,f(r)$ ,求解 $f(n)$。 $O(r-l)$
    \item 前置:逆元组合数
    \item 模数为质数
\end{itemize}
\lstinputlisting{code/math/拉格朗日插值.cpp}

%--------------------------------------------------------------------
\subsection{原根} % TODO, modify phi[n] in code
若 $g$ 满足:
\begin{equation*}
    \begin{aligned}
        (g,m)=1\\
        \delta_m(g)=\phi(m)
    \end{aligned}
\end{equation*}
则 $g$ 为 $m$ 的原根。
找所有原根:
\begin{enumerate}
    \item 找到最小的原根 $g$。如果一个数 $g$ 是原根,那么 $\forall p|\phi(m):g^{\dfrac{\phi(m)}{p}}\neq1$
    \item 找小于 $m$ 与 $\phi(m)$ 互质的数 $k$ ,则 $g^k$ 也是原根(能覆盖所有原根)个数为 $\phi(\phi(m))$ 个。
\end{enumerate}
题目:找出 $n$ 的所有原根,间隔 $d$ 输出。
\lstinputlisting{code/math/原根.cpp}
%--------------------------------------------------------------------

\subsection{Ex-Baby-Step-Giant-Step-Algorithm}

BSGS

求解 $a^x=b\pmod{p}, (0\le x< p)$

令 $x = A \left \lceil \sqrt p \right \rceil - B$,其中 $0\le A,B \le \left \lceil \sqrt p \right \rceil$,则有 $a^{A\left \lceil \sqrt p \right \rceil -B} \equiv b \pmod p$,稍加变换,则有 $a^{A\left \lceil \sqrt p \right \rceil} \equiv ba^B \pmod p$。

我们已知的是 $a,b$,所以我们可以先算出等式右边的 $ba^B$ 的所有取值,枚举 $B$,用 `hash`/`map` 存下来,然后逐一计算 $a^{A\left \lceil \sqrt p \right \rceil}$,枚举 $A$,寻找是否有与之相等的 $ba^B$,从而我们可以得到所有的 $x$,$x=A \left \lceil \sqrt p \right \rceil - B$。

注意到 $A,B$ 均小于 $\left \lceil \sqrt p \right \rceil$,所以时间复杂度为 $\Theta\left (\sqrt p\right )$,用 `map` 则多一个 $\log$。

\noindent exBSGS

其中 $a,p$ 不一定互质。

当 $a\perp p$ 时,在模 $p$ 意义下 $a$ 存在逆元,因此可以使用 BSGS 算法求解。于是我们想办法让他们变得互质。

具体地,设 $d_1=\gcd(a,p)$。如果 $d_1\nmid b$,则原方程无解。否则我们把方程同时除以 $d_1$,得到

$$ \frac{a}{d_1}\cdot a^{x-1}\equiv \frac{b}{d_1}\pmod{\frac{p}{d_1}} $$

如果 $a$ 和 $\frac{p}{d_1}$ 仍不互质就再除,设 $d_2=\gcd\left(a,\frac{p}{d_1}\right)$。如果 $d_2\nmid \frac{b}{d_1}$,则方程无解;否则同时除以 $d_2$ 得到

$$ \frac{a^2}{d_1d_2}\cdot a^{x-2}≡\frac{b}{d_1d_2} \pmod{\frac{p}{d_1d_2}} $$

同理,这样不停的判断下去。直到 $a\perp \frac{p}{d_1d_2\cdots d_k}$。

记 $D=\prod_{i=1}^kd_i$,于是方程就变成了这样:

$$ \frac{a^k}{D}\cdot a^{x-k}\equiv\frac{b}{D} \pmod{\frac{p}{D}} $$

由于 $a\perp\frac{p}{D}$,于是推出 $\frac{a^k}{D}\perp \frac{p}{D}$。这样 $\frac{a^k}{D}$ 就有逆元了,于是把它丢到方程右边,这就是一个普通的 BSGS 问题了,于是求解 $x-k$ 后再加上 $k$ 就是原方程的解啦。

注意,不排除解小于等于 $k$ 的情况,所以在消因子之前做一下 $\Theta(k)$ 枚举,直接验证 $a^i\equiv b \pmod p$,这样就能避免这种情况。

\begin{itemize}
    \item 注意,inv 必须由扩欧求!
    \item 注意开 long long
    \item 前置:ksm, exgcd求逆元
\end{itemize}
\lstinputlisting{code/math/ExBSGS.cpp}

%--------------------------------------------------------------------
\subsection{逆元}

\subsubsection{exgcd求逆元}
\begin{itemize}
    \item 前置:exgcd
    \item $(x,p)=1$
\end{itemize}
\lstinputlisting{code/math/exgcd求逆元.cpp}

\subsubsection{快速幂求逆元}

根据费马小定理:$p\in primes\to a^{-1}\equiv a^{p - 2}\pmod{p}$

\subsubsection{整数除法取模}
如果 $\dfrac{a}{b}\in \mathbb{N}$, $b\times p$ 可以在计算机中表示,那么 $\dfrac{a}{b}\bmod p=\dfrac{a\bmod(p\times b)}{b}$
%--------------------------------------------------------------------
\subsection{上下取整}
\begin{itemize}
    \item $b$ 必须为正整数。
\end{itemize}
\lstinputlisting{code/math/上下取整.cpp}
%--------------------------------------------------------------------
\subsection{线性基}
\begin{itemize}
    \item $O(\log x)$ insert
    \item $O(\log^2 x)$ get-kth
    \item $O(\log x)$ get-max
    \item 如果问能否通过选一些数(不能不选)异或得到0,必须特判。
\end{itemize}
\lstinputlisting{code/math/线性基.cpp}
%--------------------------------------------------------------------
\subsection{高斯消元}
\begin{itemize}
    \item equ 是方程个数,n 是变元个数,答案存在 ans。
    \item return : 无解(-1),自由变元个数。
\end{itemize}

\subsubsection{解异或线性方程组}
\lstinputlisting{code/math/01高斯消元.cpp}

\subsubsection{解double线性方程组}
\lstinputlisting{code/math/double高斯消元.cpp}

\subsubsection{解模意义线性方程组}
\begin{itemize}
    \item 时间复杂度 $O(n^3 \log{mod})$
\end{itemize}
\lstinputlisting{code/math/模意义高斯消元.cpp}
%--------------------------------------------------------------------
\subsection{Miller-Rabin}
\begin{itemize}
    \item 前置:快速幂(\_\_int128!!!)
    \item int 范围: {2, 7, 61}
    \item long long 范围: {2, 325, 9375, 28178, 450775, 9780504, 1795265022}
    \item 4E13: {2, 2570940, 211991001, 3749873356}
    \item 3E15: {2, 2570940, 880937, 610386380, 4130785767}
    \item 注意看判断范围是 int 还是 long long
\end{itemize}

\lstinputlisting{code/math/miller-rabin.cpp}
%--------------------------------------------------------------------
\subsection{Pollard-Rho}
\begin{itemize}
    \item 前置:Miller-Rabin
\end{itemize}

\lstinputlisting{code/math/Pollard-Rho.cpp}
%--------------------------------------------------------------------
\subsection{拆系数FFT/MTT}
\begin{itemize}
    \item FFT,内含MTT,比较灵活。
\end{itemize} 
\lstinputlisting{code/math/FFT.cpp}

%--------------------------------------------------------------------
\subsection{多项式全家桶(Number-Theoretic-Transform)}
\begin{itemize}
    \item 注意调整原根g,模数mod,N开3到4倍数据范围,附录A
    \item 注意resize()
    \item 注意Inv/Ln的时候常数项不能为0
    \item 注意Exp的时候常数项必须是0
    \item 注意这里面的 ksm() 第三个参数是初值而不是模数
\end{itemize}
\lstinputlisting{code/math/NTT.cpp}

%--------------------------------------------------------------------


\newpage
\section{Math Formula}

\subsection{多项式牛顿迭代}

\subsection{牛顿恒等式}
设 $E(x)=1+e_1x+e_2x^2+...$.

\begin{equation*}
    \begin{aligned}
        \prod_{i=1}^n (1+a_ix)&=E(x)\\
        (\ln \prod_{i=1}^n (1+a_ix))'&=E'(x)/E(x)\\
        \sum_{i=1}^n \frac{a_i}{1+a_ix}&=E'(x)/E(x)\\
        E'(x)/E(x)&=\sum_{i=1}^n\sum_{j\ge 0}(-1)^{j}a_i^{j+1}x^j\\
        &=\sum_{j\ge 0}(-1)^j\left(\sum_{i=1}^n a_i^{j+1}\right)x^j\\
        &=\sum_{j\ge 0}(-1)^j p_{j+1} x^j
    \end{aligned}
\end{equation*}

结论:
\begin{equation*}
    \begin{aligned}
        ke_k&=\sum_{i=1}^{k}(-1)^{i-1}e_{k-i}p_i\\
        0&=\sum_{i=k-n}^k(-1)^{i-1}e_{k-i}p_i
    \end{aligned}
\end{equation*}

\subsection{生成函数/形式幂级数}

\subsection{数论公式}
\subsubsection{莫比乌斯反演}
一般反演:
\begin{equation*}
    \begin{aligned}
        f(n)=\sum_{d\mid n}g(d) \Longleftrightarrow g(n)=\sum_{d\mid n} \mu (d)f(\frac{n}{d})\\
        f(n)=\sum_{\substack{n\mid d\\ d\le N}}g(d) \Longleftrightarrow g(n)=\sum_{\substack{n\mid d\\ d\le N}} \mu (\frac{d}{n})f(d)
    \end{aligned}
\end{equation*}

gcd 反演结论:
\begin{equation*}
    [\operatorname{gcd}(i, j)=1]=\sum_{d \mid \operatorname{gcd}(i, j)} \mu(d)
\end{equation*}
\subsubsection{杜教筛}
\begin{equation*}
    g(1) S(n)=\sum_{i=1}^{n}(f * g)(i)-\sum_{i=2}^{n} g(i) S\left(\left\lfloor\frac{n}{i}\right\rfloor\right)
\end{equation*}
\subsubsection{Min\_25筛/Ex-Eratosthenes-Sieve}
第一步:筛出所有质数部分(此处的$g(n)$是$f(n),n\in \mathbb{P}$的通项公式)
\begin{equation*}
    \begin{array}{c}
        G_{k}(n):=\sum_{i=1}^{n}\left[p_{k}<\operatorname{lpf}(i) \vee \operatorname{isprime}(i)\right] g(i)\\
        G_{k}(n)=G_{k-1}(n)-\left[p_{k}^{2} \leq n\right] g\left(p_{k}\right)\left(G_{k-1}\left(n / p_{k}\right)-G_{k-1}\left(p_{k-1}\right)\right)
    \end{array}
\end{equation*}
第二部:筛出质数和合数部分,我们令 $F_{\text{prime}}(n)=\sum_{\substack{2\le p\le n\\ p\in \text{prime}}}f(p)$
\begin{equation*}
    \begin{aligned} 
        F_{k}(n) &= \sum_{i = 2}^{n} [p_{k} \le \operatorname{lpf}(i)] f(i) \ \\
        &= \sum_{\substack{k \le i \\ p_{i}^{2} \le n}} \sum_{\substack{c \ge 1 \\ p_{i}^{c} \le n}} f\left(p_{i}^{c}\right) ([c > 1] + F_{i + 1}\left(n / p_{i}^{c}\right)) + \sum_{\substack{k \le i \\ p_{i} \le n}} f(p_{i}) \ \\
        &{= \sum_{\substack{k \le i \\ p_{i}^{2} \le n}} \sum_{\substack{c \ge 1 \\ p_{i}^{c} \le n}} f\left(p_{i}^{c}\right) ([c > 1] + F_{i + 1}\left(n / p_{i}^{c}\right)) + F_{\mathrm{prime}}(n) - F_{\mathrm{prime}}(p_{k - 1})}\text{常用!!} \ \\
        &= \sum_{\substack{k \le i \\ p_{i}^{2} \le n}} \sum_{\substack{c \ge 1 \\ p_{i}^{c + 1} \le n}} \left(f\left(p_{i}^{c}\right) F_{i + 1}\left(n / p_{i}^{c}\right) + f\left(p_{i}^{c + 1}\right)\right) + F_{\mathrm{prime}}(n) - F_{\mathrm{prime}}(p_{k - 1}) 
    \end{aligned}
\end{equation*}

\lstinputlisting{code/math/min25.cpp}



\subsection{分配问题}

\begin{enumerate}
    \item $n$ 个球放到 $k$ 个盒子,每个盒子只有一种形态。

    $$
    \begin{array}{cc|cc}
    \hline n \text { 个球 } & k \text { 个盒子 } & \text { 盒子可以为空 } & \text { 每个盒子内至少有一个球 } \\
    \hline \text { 有标号 } & \text { 有标号 } & k^{n} & k ! S_{2}(n, k) \\
    \text { 有标号 } & \text { 无标号 } & \sum_{i=1}^{k} S_{2}(n, i) & S_{2}(n, k) \\
    \text { 无标号 } & \text { 有标号 } & C(n+k-1, k-1) & C(n-1, k-1) \\
    \text { 无标号 } & \text { 无标号 } & p(n+k, k) & p(n, k) \\
    \hline
    \end{array}
    $$

    其中 $S_2(n,k)$ 为第二类 `Stirling数` ,$p(n,k)$ 为 `分拆数`

    \item $n$ 个球放到 $k$ 个盒子, 每个盒子至少一个球,装有 $i$ 个球的盒子有 $f_i$ ($i\ge1$) 种形态。

    $F(x) $ 是 $f_i$ 的 $o.g.f.$ ,$E(x)$ 是 $f_i$ 的 $e.g.f.$
    $$
    \begin{array}{cc|c}
    \hline n \text { 个球 } & k \text { 个盒子 } & \text { 关于 $n$ 方案的生成函数 } \\
    \hline \text { 有标号 } & \text { 有标号 } & e . g . f=E(x)^{k} \\
    \text { 有标号 } & \text { 无标号 } &  { e.g.f }=\frac{1}{k !} E(x)^{k} \\
    \text { 无标号 } & \text { 有标号 } &  { o.g.f }=F(x)^{k} \\
    \text { 无标号 } & \text { 无标号 } & \text { 不会 } \\
    \hline
    \end{array}
    $$


    \item $n$ 个球放到若干盒子,每个盒子至少一个球,装有 $i$ 个球的盒子有 $f_i$ ($i\ge1$) 种形态.

    $$
    \begin{array}{cc|c}
    \hline n \text { 个球 } & \text { 盒子 } & \text { 方案的生成函数 } \\
    \hline \text { 有标号 } & \text { 有标号 } & e . g . f=\frac{1}{1-E(x)} \\
    \text { 有标号 } & \text { 无标号 } &  { e.g.f }=\exp (E(x)) \\
    \text { 无标号 } & \text { 有标号 } &  { o.g.f }=\frac{1}{1-F(x)} \\
    \text { 无标号 } & \text { 无标号 } & o . g . f=\prod_{i \geq 1}\left(\frac{1}{1-x^{i}}\right)^{f_{i}}=\exp \left(\sum_{j \geq 1} \frac{1}{j} F\left(x^{j}\right)\right) \\
    \hline
    \end{array}
    $$
\end{enumerate}

\subsection{第二类斯特林数}

性质:
\begin{enumerate}
    \item ${n\brace k}={n-1\brace k-1}+k{n-1\brace k}$ 边界条件 ${x\neq 1\brace 0}=0,{0\brace 0}=1$
    \item ${n\brace k}=\frac{1}{k!}\sum\limits_{i=0}^{k}(-1)^i\binom{k}{i}(k-i)^{n}$ (用来求第二类斯特林数·行,考虑容斥)
    \item $k^n=\sum\limits_{i=0}^k {n\brace i} k^{\underline{i}}$ (性质 2 的反演)
    \item 重要公式:$x^n=\sum\limits_{k=0}^{n}{n\brace k}x^{\underline k}$ (基于 3,$x$ 个球 $n$ 个盒子)
    \item 关于 $n$ 的 $e.g.f.=\dfrac{(e^x-1)^k}{k!}$ (考虑将 $n$ 个物品染成 $k$ 种颜色,每一种物品的指数生成函数为 $e^x-1$ )(用于求第二类斯特林数·列)
\end{enumerate}

\subsection{第一类斯特林数}

性质:
\begin{enumerate}
    \item ${n\brack k}={n-1\brack k-1}+(n-1){n-1\brack k}$ 边界条件 ${x\neq1\brack 0}=0,{0\brack 0}=1$
    \item 重要公式:$x^{\overline n}=\sum\limits_{k=0}^{n}{n\brack k}x^n$ ($x$ 个球 $n$ 个盒子)(用于计算第一类斯特林数·行)
    \item 有符号的第一类斯特林数:$S_1(n,k)=(-1)^{n+k}{n\brack k}$
    
    $x^{\underline n}=\sum\limits_{k=0}^{n}S_1(n,k)x^n$
    \item 关于 $n$ 的 $e.g.f.=\dfrac{(-\ln(1-x))^k}{k!}$. (列)
\end{enumerate}

\subsection{分拆数}

性质:
\begin{enumerate}
    \item $p(n,k)=p(n-1,k-1)+p(n-k,k)$
    \item $o.g.f. = x^k \prod\limits_{i=1}^k \frac{1}{1-x^i}$ (考虑 Ferrers 图中长度为 $i$ 的一列有多少个)
\end{enumerate}

\subsection{五边形数}

性质:
\begin{enumerate}
    \item $p(n)=\sum\limits_{k=1}^{n}p(n,k)$
    \item $o.g.f.=\prod\limits_{i\ge1}{\frac{1}{1-x^i}}$\\
    考虑求 $\ln$ , $o.g.f.=\exp(\sum_{n\ge1}{\sum_{d\mid n} {\frac{x^n}{d}}})$ $O(n\log{n})$
    \item 递推公式:
        \begin{enumerate}
            \item 考虑 $Q(x)=\prod_{i\ge1}(1-x^i)$
            \item $Q(x)=\sum_{n\ge0}(q_{even}(n)-q_{odd}(n))x^n,其中 $ $q_{even/odd}$ 表示有偶数/奇数行,每一行的个数都不相同,大部分 $q_{even}(n)$ 和 $q_{odd}(n)$ 抵消,小部分也就是有 $b$ 行,$n=\frac{b(3b-1)}{2}$ 和 $n=\frac{b(3b+1)}{2}$ 无法抵消。其系数都是 $(-1)^b$
            \item 则 $Q(x)=1+\sum_{i\ge1}(-1)^ix^{\dfrac{(3i\pm1)i}{2}}$
            \item $P(x)=Q^{-1}(x)$ 则,$p(n)=\sum_{k\ge1}(-1)^{k-1}\left(p\left(n-\dfrac{(3k-1)k}{2}\right)+p\left(n-\dfrac{(3k+1)k}{2}\right)\right)$
        \end{enumerate}
\end{enumerate}

\subsection{Polya}

\paragraph{置换群}\

$G$ 是置换的集合,$◦$ 是置换的复合,且 $(G, ◦)$ 为一个群时,称 $(G, ◦)$ 为一个置换群。 \\


\noindent 旋转群:

设 $n$ 元环的 $n$ 个结点分别为 $a_1,a_2, ...,a_n$,旋转操作可以看成 $A = {a_1, ..., a_n}$ 上的 $n$ 个置换,其中第 $i$ 个置换为 $g_i = [i + 1,i + 2, ...,n,1, ...,i]$。 

设集合 $G= \{g_0,g_1, ...,g_{n−1}\}$,则 $(G,◦)$ 是一个置换群,称为正 $n$ 边形的旋转群。\\

\paragraph{群对集合的作用}\

一个操作会将一个对象改变为另一个对象,形式化地:

设 $(G, \circ)$ 是一个群, 其单位元为 $e, X$ 是一个集合, 群 $G$ 对集合 $X$ 的一个作用是一个 $G \times X$ 到 $X$ 的映射 $f$, 满足:

\begin{itemize}
    \item $\forall x \in X . f(e, x)=x$
    \item $\forall g, h \in G . f(h \circ g, x)=f(h, f(g, x))$
    我们把 $f(g, x)$ 简记成 $g_{f}(x)$ 或(在没有歧义的情况下记成) $g(x)$ 。
\end{itemize}

\begin{itemize}
    \item 设 $n$ 元环的 $n$ 个结点分别为 $a_{1}, a_{2}, \ldots, a_{n}$, 令 $A=\left\{a_{1}, \ldots, a_{n}\right\}$ 。设颜色集合为 $B=\left\{b_{1}, b_{2}, \ldots, b_{m}\right\}$ 。给 $n$ 元环染色可以看成 $A$ 到 $B$ 的一个映射 $x: A \rightarrow B$, 令 $X$ 是 所有这些映射的集合, 即 $X=\{x \mid x: A \rightarrow B\}$ 。
    \item 令 $G=\left\{g_{0}, g_{1}, \ldots, g_{n-1}\right\}$ 是正 $n$ 边形的旋转群, 定义 $G \times X$ 到集合 $X$ 的映射 $f$, 其中 $\forall i=0 . . n-1, x \in X . y=f\left(g_{i}, x\right)$ 满足 $\forall j=1 . . n . y\left(a_{j}\right)=x\left(g_{i}\left(a_{j}\right)\right)$, 可以证明 $f$ 是群 $G$ 到集合 $X$ 的一个作用, 因此我们简记 $y=f\left(g_{i}, x\right)$ 为 $y=g_{i}(x)$ 。
    \item $X$ 上的 $G$ 关系为 $R_{G}=\{(x, y) \mid x, y \in X \wedge(\exists g \in G . y=g(x))\}, x R_{G} y$ 当且仅当染色 方案 $x$ 能通过旋转得到 $y$ 。
\end{itemize}

要求所有不同的染色方案, 即是求 $X$ 上的 $G$-轨道的数量。\\

\paragraph{Burnside 引理}\

设有限群 $(G, \circ)$ 作用在有限集 $X$ 上, 则 $X$ 上的 $G$-轨道数量为
$$
N=\frac{1}{|G|} \sum_{g \in G} \Psi(g)
$$
其中 $\Psi(g)$ 表示 $g(x)=x$ 的 $x$ 的数量。\\

\paragraph{轮换指标}\

设 $(G, \circ)$ 是一个 $n$ 元置换的置换群, 它的轮换指标为
$$
P_{G}\left(x_{1}, x_{2}, \ldots, x_{n}\right)=\frac{1}{|G|} \sum_{g \in G} x_{1}^{b_{1}} x_{2}^{b_{2}} \ldots x_{n}^{b_{n}}
$$
$x_i^{b_i}$ 表示 $g$ 这个置换的长度为 $i$ 的环有 $b_i$ 个。

正 $n$ 边形旋转群轮换指标:

$$
P_G=\frac{1}{n}\sum_{d\mid n}\varphi(d)x_d^{n/d}
$$

正 $n$ 边形二面体群轮换指标(即可对称):

$$
P_{G}=\frac{1}{2 n} \sum_{d \mid n} \varphi(d) x_{d}^{n / d}+\left\{\begin{array}{ll}
\frac{1}{2} x_{1} x_{2}^{\frac{n-1}{2}}, \quad &n \text { 为奇数 } \\
\frac{1}{4}\left(x_{2}^{\frac{n}{2}}+x_{1}^{2} x_{2}^{\frac{n-2}{2}}\right), &n \text { 为偶数 }
\end{array}\right.
$$

正方体置换群:

顶点置换群:

$$
P_{G}=\frac{1}{24}\left(x_{1}^{8}+8 x_{1}^{2} x_{3}^{2}+9 x_{2}^{4}+6 x_{4}^{2}\right)
$$

边置换群:

$$
P_{G}=\frac{1}{24}\left(x_{1}^{12}+8 x_{3}^{4}+6 x_{1}^{2} x_{2}^{5}+3 x_{2}^{6}+6 x_{4}^{3}\right)
$$

面置换群:

$$
P_{G}=\frac{1}{24}\left(x_{1}^{6}+8 x_{3}^{2}+6 x_{2}^{3}+3 x_{1}^{2} x_{2}^{2}+6 x_{1}^{2} x_{4}\right)
$$



\paragraph{Polya 定理}\

集合 $X$ 可以看成是给集合 $A=\left\{a_{1}, a_{2}, \ldots, a_{n}\right\}$ 的每个元素赋予式样(颜色, 种类等) 的映射的集合

引入表示式样的集合 $B$, 令 $X=\{x \mid x: A \rightarrow B\}$, 记为 $B^{A}$

\textbf{式样清单}: $G$ 作用在 $B^{A}$ 上的 $G$-轨道的集合称为 $B^{A}$ 关于 $G$ 的\textbf{式样清单},记为 $F$

种类的权值: 假设 $B$ 上的每个元素 $b$ 都赋予了权值 $w(b)$

$f \in B^{A}$ 的权值: 定义 $w(f):=\prod_{a \in A} w(f(a))$

$G$-轨道的权值: $w(F):=w(f)$, 任选一个 $f \in F$

定理:

$B^{A}$ 关于 $G$ 的\textbf{式样清单}记为 $\mathcal{F}$, 则
$$
\sum_{F \in \mathcal{F}} w(F)=P_{G}\left(\sum_{b \in B} w(b), \sum_{b \in B} w(b)^{2}, \ldots, \sum_{b \in B} w(b)^{n}\right)
$$
\newpage
\section{Misc}

\subsection{德州扑克}
\lstinputlisting{code/misc/Texas_poker.cpp}

\subsection{Expression Parser}
\lstinputlisting{code/misc/expressionparser.cpp}

\subsection{原根表}
\lstinputlisting{code/misc/yuangen.cpp}

\subsection{debuger}
\lstinputlisting{code/misc/debuger.cpp}

\subsection{(补充kuangbin板子)KM算法}
\lstinputlisting{code/misc/km.cpp}

\subsection{卡题时看一眼}
\begin{itemize}
    \item meet in the middle
    \item sqrt (分块,小的暴力大的ds维护)
    \item 多源点BFS
    \item flow
    \item 容斥
    \item dp
    \item 构造
    \item 直接暴力复杂度可能是对的
    \item gcd(x, y) = gcd(x, y - x)
    \item 1e9以内因子数最多的数的因子只有1344个,1e12:6720,1e18:103680
    \item MST,DIJ
    \item 快读,longlong, \_\_int128,取模
\end{itemize}

\end{document}