\section{Math}
%--------------------------------------------------------------------
\subsection{欧拉筛/积性函数筛/线性筛}
\begin{itemize}
    \item 积性函数筛,$f(pq)=f(p)f(q),(p,q)=1$
    \item calc\_f(val, power) 返回 $f(val^{power})$
\end{itemize}
\lstinputlisting{code/math/欧拉筛.cpp}
%--------------------------------------------------------------------
\subsection{快速幂}
\lstinputlisting{code/math/ksm.cpp}

%--------------------------------------------------------------------
\subsection{组合数}
\subsubsection{暴力}
\begin{itemize}
    \item 暴力求组合数 $\binom{n}{k}$, 时间复杂度 $O(\min(k, n - k))$.
    \item 前置:快速幂
    \item 模数必须是质数!
\end{itemize}
\lstinputlisting{code/math/暴力组合数.cpp}
%--------------------------------------------------------------------
\subsubsection{递推}
\begin{itemize}
    \item $O(n^2)$ 递推求,模数随意。
\end{itemize}
\lstinputlisting[]{code/math/递推组合数.cpp}
%--------------------------------------------------------------------
\subsubsection{逆元}
\begin{itemize}
    \item 模数必须是质数!
    \item 前置:快速幂
\end{itemize}
\lstinputlisting{code/math/逆元组合数.cpp}
%--------------------------------------------------------------------
\subsection{ExGCD}
\begin{itemize}
    \item 求解 $ax+by=(a,b)$ 的特解 $x_0, y_0$.
    \item 通解$x^* = x_0+\frac{bk}{(a,b)}, y^* = y_0-\frac{ak}{(a,b)}$ ( $k \in \mathbb{Z}$ ) .
\end{itemize}
\lstinputlisting{code/math/ExGCD.cpp}
%--------------------------------------------------------------------
\subsection{拉格朗日插值}
\begin{itemize}
    \item $f(x)$是多项式,并且我们知道一系列连续的点值 $f(l),\cdots,f(r)$ ,求解 $f(n)$。 $O(r-l)$
    \item 前置:逆元组合数
    \item 模数为质数
\end{itemize}
\lstinputlisting{code/math/拉格朗日插值.cpp}

%--------------------------------------------------------------------
\subsection{原根} % TODO, modify phi[n] in code
若 $g$ 满足:
\begin{equation*}
    \begin{aligned}
        (g,m)=1\\
        \delta_m(g)=\phi(m)
    \end{aligned}
\end{equation*}
则 $g$ 为 $m$ 的原根。
找所有原根:
\begin{enumerate}
    \item 找到最小的原根 $g$。如果一个数 $g$ 是原根,那么 $\forall p|\phi(m):g^{\dfrac{\phi(m)}{p}}\neq1$
    \item 找小于 $m$ 与 $\phi(m)$ 互质的数 $k$ ,则 $g^k$ 也是原根(能覆盖所有原根)个数为 $\phi(\phi(m))$ 个。
\end{enumerate}
题目:找出 $n$ 的所有原根,间隔 $d$ 输出。
\lstinputlisting{code/math/原根.cpp}
%--------------------------------------------------------------------

\subsection{Ex-Baby-Step-Giant-Step-Algorithm}

BSGS

求解 $a^x=b\pmod{p}, (0\le x< p)$

令 $x = A \left \lceil \sqrt p \right \rceil - B$,其中 $0\le A,B \le \left \lceil \sqrt p \right \rceil$,则有 $a^{A\left \lceil \sqrt p \right \rceil -B} \equiv b \pmod p$,稍加变换,则有 $a^{A\left \lceil \sqrt p \right \rceil} \equiv ba^B \pmod p$。

我们已知的是 $a,b$,所以我们可以先算出等式右边的 $ba^B$ 的所有取值,枚举 $B$,用 `hash`/`map` 存下来,然后逐一计算 $a^{A\left \lceil \sqrt p \right \rceil}$,枚举 $A$,寻找是否有与之相等的 $ba^B$,从而我们可以得到所有的 $x$,$x=A \left \lceil \sqrt p \right \rceil - B$。

注意到 $A,B$ 均小于 $\left \lceil \sqrt p \right \rceil$,所以时间复杂度为 $\Theta\left (\sqrt p\right )$,用 `map` 则多一个 $\log$。

\noindent exBSGS

其中 $a,p$ 不一定互质。

当 $a\perp p$ 时,在模 $p$ 意义下 $a$ 存在逆元,因此可以使用 BSGS 算法求解。于是我们想办法让他们变得互质。

具体地,设 $d_1=\gcd(a,p)$。如果 $d_1\nmid b$,则原方程无解。否则我们把方程同时除以 $d_1$,得到

$$ \frac{a}{d_1}\cdot a^{x-1}\equiv \frac{b}{d_1}\pmod{\frac{p}{d_1}} $$

如果 $a$ 和 $\frac{p}{d_1}$ 仍不互质就再除,设 $d_2=\gcd\left(a,\frac{p}{d_1}\right)$。如果 $d_2\nmid \frac{b}{d_1}$,则方程无解;否则同时除以 $d_2$ 得到

$$ \frac{a^2}{d_1d_2}\cdot a^{x-2}≡\frac{b}{d_1d_2} \pmod{\frac{p}{d_1d_2}} $$

同理,这样不停的判断下去。直到 $a\perp \frac{p}{d_1d_2\cdots d_k}$。

记 $D=\prod_{i=1}^kd_i$,于是方程就变成了这样:

$$ \frac{a^k}{D}\cdot a^{x-k}\equiv\frac{b}{D} \pmod{\frac{p}{D}} $$

由于 $a\perp\frac{p}{D}$,于是推出 $\frac{a^k}{D}\perp \frac{p}{D}$。这样 $\frac{a^k}{D}$ 就有逆元了,于是把它丢到方程右边,这就是一个普通的 BSGS 问题了,于是求解 $x-k$ 后再加上 $k$ 就是原方程的解啦。

注意,不排除解小于等于 $k$ 的情况,所以在消因子之前做一下 $\Theta(k)$ 枚举,直接验证 $a^i\equiv b \pmod p$,这样就能避免这种情况。

\begin{itemize}
    \item 注意,inv 必须由扩欧求!
    \item 注意开 long long
    \item 前置:ksm, exgcd求逆元
\end{itemize}
\lstinputlisting{code/math/ExBSGS.cpp}

%--------------------------------------------------------------------
\subsection{逆元}

\subsubsection{exgcd求逆元}
\begin{itemize}
    \item 前置:exgcd
    \item $(x,p)=1$
\end{itemize}
\lstinputlisting{code/math/exgcd求逆元.cpp}

\subsubsection{快速幂求逆元}

根据费马小定理:$p\in primes\to a^{-1}\equiv a^{p - 2}\pmod{p}$

\subsubsection{整数除法取模}
如果 $\dfrac{a}{b}\in \mathbb{N}$, $b\times p$ 可以在计算机中表示,那么 $\dfrac{a}{b}\bmod p=\dfrac{a\bmod(p\times b)}{b}$
%--------------------------------------------------------------------
\subsection{上下取整}
\begin{itemize}
    \item $b$ 必须为正整数。
\end{itemize}
\lstinputlisting{code/math/上下取整.cpp}
%--------------------------------------------------------------------
\subsection{线性基}
\begin{itemize}
    \item $O(\log x)$ insert
    \item $O(\log^2 x)$ get-kth
    \item $O(\log x)$ get-max
    \item 如果问能否通过选一些数(不能不选)异或得到0,必须特判。
\end{itemize}
\lstinputlisting{code/math/线性基.cpp}
%--------------------------------------------------------------------
\subsection{高斯消元}
\begin{itemize}
    \item equ 是方程个数,n 是变元个数,答案存在 ans。
    \item return : 无解(-1),自由变元个数。
\end{itemize}

\subsubsection{解异或线性方程组}
\lstinputlisting{code/math/01高斯消元.cpp}

\subsubsection{解double线性方程组}
\lstinputlisting{code/math/double高斯消元.cpp}

\subsubsection{解模意义线性方程组}
\begin{itemize}
    \item 时间复杂度 $O(n^3 \log{mod})$
\end{itemize}
\lstinputlisting{code/math/模意义高斯消元.cpp}
%--------------------------------------------------------------------
\subsection{Miller-Rabin}
\begin{itemize}
    \item 前置:快速幂(\_\_int128!!!)
    \item int 范围: {2, 7, 61}
    \item long long 范围: {2, 325, 9375, 28178, 450775, 9780504, 1795265022}
    \item 4E13: {2, 2570940, 211991001, 3749873356}
    \item 3E15: {2, 2570940, 880937, 610386380, 4130785767}
    \item 注意看判断范围是 int 还是 long long
\end{itemize}

\lstinputlisting{code/math/miller-rabin.cpp}
%--------------------------------------------------------------------
\subsection{Pollard-Rho}
\begin{itemize}
    \item 前置:Miller-Rabin
\end{itemize}

\lstinputlisting{code/math/Pollard-Rho.cpp}
%--------------------------------------------------------------------
\subsection{拆系数FFT/MTT}
\begin{itemize}
    \item FFT,内含MTT,比较灵活。
\end{itemize} 
\lstinputlisting{code/math/FFT.cpp}

%--------------------------------------------------------------------
\subsection{多项式全家桶(Number-Theoretic-Transform)}
\begin{itemize}
    \item 注意调整原根g,模数mod,N开3到4倍数据范围,附录A
    \item 注意resize()
    \item 注意Inv/Ln的时候常数项不能为0
    \item 注意Exp的时候常数项必须是0
    \item 注意这里面的 ksm() 第三个参数是初值而不是模数
\end{itemize}
\lstinputlisting{code/math/NTT.cpp}

\begin{itemize}
    \item 简约版全家桶:
\end{itemize}
\lstinputlisting{code/math/简约全家桶.cpp}
%--------------------------------------------------------------------

